\documentclass[12pt]{scrartcl}
\usepackage[german, ngerman]{babel}
\usepackage{graphicx}
\usepackage{color}
\usepackage{url}
\usepackage{xcolor}
\usepackage{listings}
\usepackage{hyperref}
\usepackage{nameref}
\usepackage{varioref}
\hypersetup{
    colorlinks=true,
    linkcolor={black!50!black},
    % linkcolor={red!50!black},
    citecolor={black!50!black},
    urlcolor={black!50!black}
}
\usepackage[headsepline,footsepline]{scrlayer-scrpage}
\usepackage{biblatex}
\usepackage{amsmath}
\usepackage{float}

\newcommand{\code}[1]{\texttt{#1}}


\definecolor{mGreen}{rgb}{0,0.6,0}
\definecolor{mGray}{rgb}{0.5,0.5,0.5}
\definecolor{mPurple}{rgb}{0.58,0,0.82}
\definecolor{backgroundColour}{rgb}{0.95,0.95,0.95} %{cmyk}{0.05,0.05,0.05,0.05}

\lstdefinestyle{CStyle}{
    backgroundcolor=\color{backgroundColour},
    commentstyle=\color{mGreen},
    keywordstyle=\color{blue},
    numberstyle=\tiny\color{mGray},
    stringstyle=\color{mPurple},
    basicstyle=\footnotesize,
    breakatwhitespace=false,
    breaklines=true,
    captionpos=b,
    keepspaces=true,
    numbers=left,
    numbersep=5pt,
    showspaces=false,
    showstringspaces=false,
    showtabs=false,
    tabsize=2,
    language=C++
}

\lstdefinestyle{Terminal}{
    backgroundcolor=\color{backgroundColour},
    commentstyle=\color{black},
    keywordstyle=\color{black},
    numberstyle=\tiny\color{black},
    stringstyle=\color{black},
    basicstyle=\footnotesize,
    breakatwhitespace=false,
    breaklines=true,
    captionpos=b,
    keepspaces=true,
    numbers=none,
    numbersep=5pt,
    showspaces=false,
    showstringspaces=false,
    showtabs=false,
    tabsize=2,
}


\pagestyle{scrheadings}
\clearscrheadfoot
%\cfoot{Tobias Gruber}
\cfoot{\pagemark}
\chead{\headmark}
\automark[subsection]{section}


\begin{document}


\begin{titlepage}
    \vfill
	\centering
    \vspace{1.5cm}

	{\scshape\LARGE Hochschule München \par}
    {\scshape\Large Fakultät für Informatik und Mathematik\par}
	\vspace{1.5cm}




    \vfill
    {\LARGE\bfseries Praktikumsaufgabe 3 \\}
    \vspace{0.5cm}
	{in der Vorlesung\\}
    \vspace{0.5cm}
    {\LARGE\bfseries Computational Geometry\\~\\ \par}
	{\LARGE Bestimmung von Schnittpunkten aus einem gegebenen Satz von Strecken mit Hilfe des \\Sweep Line-Algorithmus\\~\\ \par}
	\vfill
    \vfill


    \begin{tabular}{ll}
    \normalsize
    Team:  & Christopher Hinz, Tobias Gruber\\
    Studiengruppe: & Master Informatik\\
    Studiensemester: & 1. Semester\\
    Schwerpunkt: & Embedded Computing\\
    \end{tabular}
    \vspace{1.5cm}

    11.04.2022

    \vspace{0.5cm}

    Sommersemester 2022

	\vfill

\end{titlepage}

\newpage

% table of contents
\thispagestyle{empty}
\tableofcontents
\newpage

%%%%%%%%%%%%%%%%%%%%%%%%%%%%%
% Problemstellung
%%%%%%%%%%%%%%%%%%%%%%%%%%%%%
\section{Problemstellung}

\section{Einführung}


\section{Methoden}


\section{Ergebnisse}

Neue Datei für P3: 
\begin{itemize}
    \item s\_1000\_10.dat: \\Schnittpunkte: 789, Laufzeit: 82.164 ms
\end{itemize}

Zur Überprüfung wurden einzelne Punkte getestet und insbesondere das Ergebnis mit matplotlib geplottet. Dies ist bei 1000 Strecken noch recht praktikabel.\\
Zu sehen sind alle 1000 Strecken und jeder gefundene Schnittpunkt ist als Punkt dargestellt.\\
\begin{figure}[ht]
    \advance\leftskip-4cm
    \includegraphics[scale=0.48]{Plot_all.png}
\end{figure}

Durch Vergrößern bestimmter Bereiche können die Resultate einfach und sicher überprüft werden.\\
\begin{figure}[ht]
    \advance\leftskip-4cm
    \includegraphics[scale=0.48]{Plot_zoom.png}
\end{figure}


Dateien aus P1:\\
\begin{itemize}
    \item s\_1000\_1.dat:   \\reduziert auf 997, Schnittpunkte: 4, Laufzeit: 38.592 ms
    \item s\_10000\_1.dat:  \\reduziert auf 9997, Schnittpunkte: 725, Laufzeit: 1956.78 ms
    \item s\_100000\_1.dat: \\reduziert auf 99985, Schnittpunkte: 76087, Laufzeit: 556070 ms
\end{itemize}


\begin{lstlisting}[style=Terminal, caption={testing.cpp: Ausgabe Konsole},captionpos=b, label={lst:ausgabe_test}]
\end{lstlisting}



\end{document}
