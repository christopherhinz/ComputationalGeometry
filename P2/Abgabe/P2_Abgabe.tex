\documentclass[12pt]{scrartcl}
\usepackage[german, ngerman]{babel}
\usepackage{graphicx}
\usepackage{color}
\usepackage{url}
\usepackage{xcolor}
\usepackage{listings}
\usepackage{hyperref}
\usepackage{nameref}
\usepackage{varioref}

\usepackage[headsepline,footsepline]{scrlayer-scrpage}
\usepackage{biblatex}
\usepackage{amsmath}
\usepackage{float}

\newcommand{\code}[1]{\texttt{#1}}


\definecolor{mGreen}{rgb}{0,0.6,0}
\definecolor{mGray}{rgb}{0.5,0.5,0.5}
\definecolor{mPurple}{rgb}{0.58,0,0.82}
\definecolor{backgroundColour}{rgb}{0.95,0.95,0.95} %{cmyk}{0.05,0.05,0.05,0.05}

\lstdefinestyle{CStyle}{
    backgroundcolor=\color{backgroundColour},
    commentstyle=\color{mGreen},
    keywordstyle=\color{blue},
    numberstyle=\tiny\color{mGray},
    stringstyle=\color{mPurple},
    basicstyle=\footnotesize,
    breakatwhitespace=false,
    breaklines=true,
    captionpos=b,
    keepspaces=true,
    numbers=left,
    numbersep=5pt,
    showspaces=false,
    showstringspaces=false,
    showtabs=false,
    tabsize=2,
    language=C++
}

\lstdefinestyle{Terminal}{
    backgroundcolor=\color{backgroundColour},
    commentstyle=\color{black},
    keywordstyle=\color{black},
    numberstyle=\tiny\color{black},
    stringstyle=\color{black},
    basicstyle=\footnotesize,
    breakatwhitespace=false,
    breaklines=true,
    captionpos=b,
    keepspaces=true,
    numbers=none,
    numbersep=5pt,
    showspaces=false,
    showstringspaces=false,
    showtabs=false,
    tabsize=2,
}


\pagestyle{scrheadings}
\clearscrheadfoot
%\cfoot{Tobias Gruber}
\cfoot{\pagemark}
\chead{\headmark}
\automark[subsection]{section}


\begin{document}


\begin{titlepage}
    \vfill
	\centering
	{\scshape\LARGE Hochschule München \par}
    {\scshape\Large Fakultät für Informatik \par}
	\vspace{1.5cm}




    \vfill
	{\LARGE\bfseries Computational Geometry\\~\\ \par}
	{\LARGE\bfseries Praktikumsaufgabe 2\par}
	\vfill
    \vfill


    \begin{tabular}{ll}
    \normalsize
    Team:  & Christopher Hinz, Tobias Gruber\\
    \end{tabular}

	\vfill

\end{titlepage}

\newpage



\raggedright

%%%%%%%%%%%%%%%%%%%%%%%%%%%%%
% Problemstellung
%%%%%%%%%%%%%%%%%%%%%%%%%%%%%
\section{Problemstellung}
Lesen Sie die SVG-Datei 'DeutschlandMitStaedten.svg' und ermitteln Sie die Flächen der einzelnen Bundesländer (bezüglich der in der Datei verwendeten Skala).
Am Ende der Datei befinden sich Koordinaten von Städten, Versuchen Sie herauszufinden (bzw. lassen Sie das Ihren Rechner machen ;-), in welchem Bundesland diese jeweils liegen.\\~\\
Auch für diese Aufgabe sollten Sie versuchen, mich in der Ausarbeitung zu überzeugen (bzw. mir möglichst einfach nachvollziehbar machen), warum die Flächen, die Sie ermittelt haben, korrekt sind.


%%%%%%%%%%%%%%%%%%%%%%%%%%%%%
% Umsetzung
%%%%%%%%%%%%%%%%%%%%%%%%%%%%%
\section{Umsetzung}

Zum Einlesen der svg-Datei wurde auf die Parser-Funktionen der RapidXML Library zurückgeriffen.

\ \\~\\

Zur Überprüfung ob alle Koordinaten für die entsprechenden Bundesländer und Städte korrekt eingelesen wurden dient ein Python-Skript.
Hierfür schreibt das C++-Programm für alle Bundesländer die gefundenen Polygone (teilweise mehr als eins da Bundesland-Gebiet aus mehreren unabhängigen Flächen besteht) in eine txt-Datei.
Dasselbe gilt für die Koordinaten der Städte, die im Zuge des Praktikums im jeweiligen Bundesland verortet werden soll.\\

Das Pyton-Skript liest alle notwendigen Informationen aus dieser txt-Datei und erstellt ein Plot. Aus diesem kann zum einen bestimmt werden ob alle Bundesländer ihre charakteristischen Umrisse besitzen
und zum anderen ob die Städte innerhalb der deutschen Grenzen liegen.\\
Nachfolgende Abbildung zeigt den, mittels matplotlib, generierten Plot.

\begin{figure}[ht]
    \centering
    \includegraphics[scale=0.5]{Übersicht.png}
    \caption{Übersichts-Plot: Bundesländer und Städte}
\end{figure}

\section{Ergebnisse}




\end{document}
