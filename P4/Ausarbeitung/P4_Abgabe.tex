\documentclass[12pt]{scrartcl}
\usepackage[german, ngerman]{babel}
\usepackage{graphicx}
\usepackage{color}
\usepackage{url}
\usepackage{xcolor}
\usepackage{listings}
\usepackage{hyperref}
\usepackage{nameref}
\usepackage{varioref}
\hypersetup{
    colorlinks=true,
    linkcolor={black!50!black},
    % linkcolor={red!50!black},
    citecolor={black!50!black},
    urlcolor={black!50!black}
}
\usepackage[headsepline,footsepline]{scrlayer-scrpage}
\usepackage{biblatex}
\usepackage{amsmath}
\usepackage{float}
\usepackage{multirow}


\newcommand{\code}[1]{\texttt{#1}}


\definecolor{mGreen}{rgb}{0,0.6,0}
\definecolor{mGray}{rgb}{0.5,0.5,0.5}
\definecolor{mPurple}{rgb}{0.58,0,0.82}
\definecolor{backgroundColour}{rgb}{0.95,0.95,0.95} %{cmyk}{0.05,0.05,0.05,0.05}

\lstdefinestyle{CStyle}{
    backgroundcolor=\color{backgroundColour},
    commentstyle=\color{mGreen},
    keywordstyle=\color{blue},
    numberstyle=\tiny\color{mGray},
    stringstyle=\color{mPurple},
    basicstyle=\footnotesize,
    breakatwhitespace=false,
    breaklines=true,
    captionpos=b,
    keepspaces=true,
    numbers=left,
    numbersep=5pt,
    showspaces=false,
    showstringspaces=false,
    showtabs=false,
    tabsize=2,
    language=C++
}

\lstdefinestyle{Terminal}{
    backgroundcolor=\color{backgroundColour},
    commentstyle=\color{black},
    keywordstyle=\color{black},
    numberstyle=\tiny\color{black},
    stringstyle=\color{black},
    basicstyle=\footnotesize,
    breakatwhitespace=false,
    breaklines=true,
    captionpos=b,
    keepspaces=true,
    numbers=none,
    numbersep=5pt,
    showspaces=false,
    showstringspaces=false,
    showtabs=false,
    tabsize=2,
}


\pagestyle{scrheadings}
\clearscrheadfoot
%\cfoot{Tobias Gruber}
\cfoot{\pagemark}
\chead{\headmark}
\automark[subsection]{section}


\begin{document}


\begin{titlepage}
    \vfill
	\centering
    \vspace{1.5cm}

	{\scshape\LARGE Hochschule München \par}
    {\scshape\Large Fakultät für Informatik und Mathematik\par}
	\vspace{1.5cm}




    \vfill
    {\LARGE\bfseries Praktikumsaufgabe 4 \\}
    \vspace{0.5cm}
	{in der Vorlesung\\}
    \vspace{0.5cm}
    {\LARGE\bfseries Computational Geometry\\~\\ \par}
	{\LARGE Konvexe Hüllen mit qhull\\~\\ \par}
	\vfill
    \vfill


    \begin{tabular}{ll}
    \normalsize
    Team:  & Christopher Hinz, Tobias Gruber\\
    Studiengruppe: & Master Informatik\\
    Studiensemester: & 1. Semester\\
    Schwerpunkt: & Embedded Computing\\
    \end{tabular}
    \vspace{1.5cm}

    \today

    \vspace{0.5cm}

    Sommersemester 2022

	\vfill

\end{titlepage}

\newpage

%%%%%%%%%%%%%%%%%%%%%%%%%%%%%
% Einführung
%%%%%%%%%%%%%%%%%%%%%%%%%%%%%
\section{Einführung}
Installieren Sie das Programm qhull, erzeugen Sie zufällige Punktmengen und berechnen Sie mit qhull konvexe Hüllen, auch in höheren Dimensionen (qhull bringt ein Werkzeug zur Erzeugung von Punktmengen mit). Plotten Sie die Zeiten für zunehmende Punktanzahlen bei unterschiedlichen Dimensionen (2-8). Versuchen Sie, die Ausgaben von qhull bei "geschwätzigster" Einstellung nachzuvollziehen, zu verstehen und ggf. mit Inhalten dieser Lehrveranstaltung in Einklang zu bringen.


%%%%%%%%%%%%%%%%%%%%%%%%%%%%%
% Funktionen von qhull
%%%%%%%%%%%%%%%%%%%%%%%%%%%%%
\section{Funktionen von qhull}

Qhull bietet 6 verschiedene Programme an:
\begin{itemize}
    \setlength\itemsep{0em}
    \item qconvex: convex hulls
    \item qdelaunay: Delaunay triangulations and furthest-site Delaunay triangulations
    \item qhalf: halfspace intersections about a point
    \item qhull: all structures with additional options
    \item qvoronoi: Voronoi diagrams and furthest-site Voronoi diagrams
    \item rbox: generate point distributions for qhull
\end{itemize}

Im Zuge dieses Praktikums werden wir sowohl rbox, zur Erzeugung von Punktmengen nutzen, als auch qconvex, zur Berechnen der konvexe Hüllen, nutzen.



\subsection{rbox (http://www.qhull.org/html/rbox.htm)}
Das Programm rbox generiert zufällige oder reguläre Punkte. Standardmäßig innerhalb eines Würfels (es sei denn die Optionen 's', 'x', oder 'y' werden übergeben).
Einige Beispiele zur Erzeugung von Punktmengen sind:
\begin{itemize}
    \setlength\itemsep{0em}
    \item rbox 10 D3: erzeugt 10 Punkte in 3D
    \item rbox 15 D4: erzeugt 15 Punkte in 4D
    \item rbox 10 D2: erzeugt 10 Punkte auf einem 2D Kreis
    \item rbox 100 W0: erzeugt 100 Punkte  auf der Oberfläche eines Würfels
\end{itemize}

\subsection{Beispiele}
Bevor auf Details der Berechnungen von qconvex eingegangen wird, sollen zwei kurze Beispiele die Eingabedaten und die Ergebnisse der Berechnung visualisieren.\\
Zur Demonstration der Funktion von qconvex sollen Zufallsdaten und deren konvexe Hülle visualisiert werden. Hierzu wurde ein Python-Skript implementiert das die Datenpunkte plottet und aus den Ergebnissen von qconvex die Hyperebenen zeichnet. Die Visualisierung kann entweder auf Basis der Normalenvektoren der Hyperebene (Option 'n') oder mit Hilfe der Vertices der Facetten (Option 'o') erfolgen

\ \\
\underline{Punkte und konvexe Hülle in 2 Dimensionen}\ \\
Im erste Beispiel werden die Datenpunkte in 2 Dimensionen erzeugt und eingelesen und deren Hyperebenen (Geraden) mit qconvex berechnet. Hierbei wurde die qconvex Option zur Ausgabe der Vertices genutzt. Mit diesen lassen sich die Hyperebenen (Geraden) zeichnen. Der nachfolgende Plot zeigt dies:
\begin{figure}[ht]
    \centering
    \includegraphics[scale=0.3]{2D_plot.png}
\end{figure}
\ \\
\underline{Punkte und konvexe Hülle in 3 Dimensionen}\ \\
Im zweiten Beispiel werden die Datenpunkte in 3 Dimensionen erzeugt und eingelesen und ebenfalls die Hyperebenen (Ebenen) mit qconvex berechnet. Hier wurde ebenfalls die berechneten Vertices genutzt. Auch dies ist untenstehend mit matplotlib visualisiert:
\begin{figure}[ht]
    \centering
    \includegraphics[scale=0.3]{cubeplot.png}
\end{figure}

\subsection{Zeitmessung}
Es soll die Laufzeit des Algorithmus für verschiedene Punktmengen in unterschiedlichen Dimensionen gemessen werden.  Eine Zeitangabe ist bei der Ausgabe der Zusammenfassung des Ergebnisses enthalten. Die Zusammenfassung erhält man durch die qconvex Option 's'. Das Ergebnis der Messungen ist untenstehend grafisch und tabellarisch dargestellt.
\begin{center}
\begin{tabular}{||c | c c c c c c c||} 
    \hline
    Dimension & 2D &       3D &      4D &       5D &       6D &       7D &       8D      \\
    \hline \hline
    \multirow{4}{*}{Laufzeit in s} & 2.3e-05 &  3.4e-05 & 3.5e-05 &  4.2e-05 &  5.8e-05 &  6.2e-05 &  4.8e-05 \\ \cline{2-8}
                                   & 3e-05 &    6.3e-05 & 0.000186 & 0.000832 & 0.003005 & 0.009608 & 0.03245 \\ \cline{2-8}
                                   & 3.6e-05 &  7.8e-05 & 0.000298 & 0.001947 & 0.009298 & 0.04809 &  0.2151  \\ \cline{2-8}
                                   & 0.000116 & 0.00021 & 0.000907 & 0.007799 & 0.06736 &  0.876 &    9.933   \\ \hline
\end{tabular}
\end{center}

\begin{figure}[ht]
    \centering
    \includegraphics[scale=0.32]{runtimes.png}
\end{figure}


\subsection{qconvex (http://www.qhull.org/html/qconvex.htm)}









\ \\~\\~\\~\\~\\~\\~\\~\\~\\~\\

Konvexe Hülle berechnen:

\begin{itemize}
    \item qconvex s: s = print summary
\end{itemize}
\ \\


Punktmengen erzeugen und Konvexe Hülle plotten:
\begin{itemize}
    \item rbox 10 D3 | qconvex s: 10 Punkte in 3D und konvexe Hülle
\end{itemize}


TODO:
\begin{itemize}
    \item Source code: https://github.com/qhull/qhull
\end{itemize}




%\begin{figure}[ht]
%    \centering
%    \includegraphics[scale=0.25]{Plot_zoom_and_all.jpeg}
%\end{figure}

%\begin{lstlisting}[style=Terminal, caption={testing.cpp: Ausgabe Konsole},captionpos=b, label={lst:ausgabe_test}]
%\end{lstlisting}



\end{document}

